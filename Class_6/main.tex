% Copyright 2004 by Till Tantau <tantau@users.sourceforge.net>.
%
% In principle, this file can be redistributed and/or modified under
% the terms of the GNU Public License, version 2.
%
% However, this file is supposed to be a template to be modified
% for your own needs. For this reason, if you use this file as a
% template and not specifically distribute it as part of a another
% package/program, I grant the extra permission to freely copy and
% modify this file as you see fit and even to delete this copyright
% notice. 

\documentclass{beamer}
\usepackage[utf8]{inputenc}
\usepackage[brazilian]{babel}

% There are many different themes available for Beamer. A comprehensive
% list with examples is given here:
% http://deic.uab.es/~iblanes/beamer_gallery/index_by_theme.html
% You can uncomment the themes below if you would like to use a different
% one:
%\usetheme{AnnArbor}
%\usetheme{Antibes}
%\usetheme{Bergen}
%\usetheme{Berkeley}
%\usetheme{Berlin}
%\usetheme{Boadilla}
%\usetheme{boxes}
\usetheme{CambridgeUS}
%\usetheme{Copenhagen}
%\usetheme{Darmstadt}
%\usetheme{default}
%\usetheme{Frankfurt}
%\usetheme{Goettingen}
%\usetheme{Hannover}
%\usetheme{Ilmenau}
%\usetheme{JuanLesPins}
%\usetheme{Luebeck}
%\usetheme{Madrid}
%\usetheme{Malmoe}
%\usetheme{Marburg}
%\usetheme{Montpellier}
%\usetheme{PaloAlto}
%\usetheme{Pittsburgh}
%\usetheme{Rochester}
%\usetheme{Singapore}
%\usetheme{Szeged}
%\usetheme{Warsaw}

\title{Aula 6 - Unix}

% A subtitle is optional and this may be deleted
\subtitle{Curso de Unix}

\author{PET Computa\c{c}ão}
% - Give the names in the same order as the appear in the paper.
% - Use the \inst{?} command only if the authors have different
%   affiliation.

\institute[UFSC] % (optional, but mostly needed)
{
%
  Departamento de Informática e Estatística\\
  Universidade de Santa Catarina}
% - Use the \inst command only if there are several affiliations.
% - Keep it simple, no one is interested in your street address.

\date{PET Computa\c{c}ão, 2015}
% - Either use conference name or its abbreviation.
% - Not really informative to the audience, more for people (including
%   yourself) who are reading the slides online

\subject{Curso de Unix}
% This is only inserted into the PDF information catalog. Can be left
% out. 

% If you have a file called "university-logo-filename.xxx", where xxx
% is a graphic format that can be processed by latex or pdflatex,
% resp., then you can add a logo as follows:

% \pgfdeclareimage[height=0.5cm]{university-logo}{university-logo-filename}
% \logo{\pgfuseimage{university-logo}}

% Delete this, if you do not want the table of contents to pop up at
% the beginning of each subsection:
\AtBeginSubsection[]
{
  \begin{frame}<beamer>{Sumário}
    \tableofcontents[currentsection,currentsubsection]
  \end{frame}
}

% Let's get started
\begin{document}

\begin{frame}
  \titlepage
\end{frame}

\begin{frame}{Sumário}
  \tableofcontents
  % You might wish to add the option [pausesections]
\end{frame}

% Section and subsections will appear in the presentation overview
% and table of contents.
\section{Outros comandos interessantes}

\subsection{df}

\begin{frame}{df}
  \begin{itemize}
  \item {O comando \textbf{df} retorna ao usuário a quantidade de espaço no disco.
  }
  \item{Para mais fácil compreensão é possível usar a flag \textbf{-h} para que o retorno esteja em unidades mais comuns ao uso humano.}
  
  \end{itemize}
\end{frame}

\subsection{du}
\begin{frame}{du}
  \begin{itemize}
  \item { O comando \textbf{du} retorna ao usuário o espaço de disco ocupado individualmente por \textbf{todos} os arquivos e sub-diretórios contidos no diretório atual.}
    \item{Usando a flag \textbf{-s} o comando irá retornar um resumo do espaço ocupado pelos diretórios indicados.
    }
    \item{\textbf{du -s Documentos} irá retornar o espaço ocupado pelo diretório documentos, enquanto \textbf{du} irá retornar o espaço ocupado por cada arquivo, sub-diretório pelo diretório atual.}
    \end{itemize}
\end{frame}



\subsection{gzip}
\begin{frame}{Comprimir Arquivo}{gzip}
  \begin{itemize}
  \item {O comando \textbf{gzip} comprime um arquivo para poupar espaço de disco.
  } 
    \item{ \textbf{gzip arquivo.txt} irá criar um \textbf{arquivo.txt.gz} que terá um tamanho menor que o anterior}
    \item{você pode usar o comando gunzip para fazer o caminho contrário.
  }
  \end{itemize}
\end{frame}

\subsection{zcat}
\begin{frame}{Concatenar/exibir arquivos comprimidos}{zcat}
  \begin{itemize}
  \item { Usa-se o zcat para ler arquivos sem ter que descompacta-los.
} 
    \item{ \textbf{zcat arquivo1.gz} irá exibir o conteudo do arquivo1.}
    \item{O mesmo comando pode ser usado para exibir diversos arquivos ao mesmo tempo, como: \textbf{zcat arquivo1 arquivo2}}
 
  \end{itemize}
\end{frame}


\subsection{file}
\begin{frame}{Mostrar tipo de arquivo}{file}
  \begin{itemize}
  \item {   
  O comando \textbf{file} retorna ao usuário o tipo de todos os arquivos indicados pelo mesmo.
  } 
  \end{itemize}
\end{frame}

\subsection{diff}
\begin{frame}{Diferença entre arquivos}{dif}
  \begin{itemize}
  \item {   
  O comando \textbf{diff} mostra a diferença entre dois arquivos.
  } 
  \item{Se a comparação não for entre textos, o comando somente dirá se os arquivos são iguais ou diferentes}
  \end{itemize}
\end{frame}

\subsection{find}
\begin{frame}{Busca}{find}
    \begin{itemize}
    \item{O comando \textbf{find} pode ser usado para encontrar um arquivo utilizando uma palavra, data, tamanho ou outros atributos que você especifique}
    \item{O comando é case sensitive, por tanto usa-se a flag \textbf{-i} para ignorar tal caracteristica.}
    \item{Para mais flags e funções específicas da função, olhe o manual da mesma com o comando \textbf{man find}}
    \end{itemize}

\end{frame}

\subsection{history}
\begin{frame}{Histórico de comandos}{history}
  \begin{itemize}
  \item {O comando \textbf{history} mostra uma lista dos comandos usados anteriormente.
  } 
    \item{ É possível utilizar o comando \textbf{!!} para reutilizar o ultimo comando dado}
    \item{ Também pode-se usar comandos como \textbf{!less} para usar o ultimo comando iniciado com \textbf{less}, \textbf{!5} para reutilizar o comando número 5 na lista do history ou \textbf{!-3} para utilizar o terceiro comando mais recente.}
  \end{itemize}
\end{frame}

\section*{Sumário dos comandos}

\begin{frame}{Sumário dos comandos}
\begin{center}
 \begin{tabular}{|| c | p{7cm}||} 
 \hline
 \textbf{Comando} & \textbf{Descri\c{c}ão}\\ [0.5ex] 
 \hline\hline
 df & Mostra a quantidade de espa\c{c}o que há no disco\\
 \hline
 du & Mostra o espa\c{c}o ocupado no disco pelo diretório atual, pelos seus arquivos e subdiretórios\\
 \hline
 gzip & Comprime o arquivo para a extensão gz, usa-se \textbf{gunzip}\\
 \hline
 zcat & Exibe o conteúdo de um arquivo comprimido sem precisar descomprimi-lo\\
 \hline
 file & Retornao tipo do arquivo especificado\\
 \hline
 diff & Exibe a diferen\c{c}a entre dois arquivos\\
 \hline
 find & Procura arquivos com a palavra especificada\\
 \hline
 history & Lista o histórico dos comandos digitados no terminal\\
 \hline
\end{tabular}
\end{center}
\end{frame}


\end{document}


