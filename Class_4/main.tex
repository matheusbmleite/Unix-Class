% Copyright 2004 by Till Tantau <tantau@users.sourceforge.net>.
%
% In principle, this file can be redistributed and/or modified under
% the terms of the GNU Public License, version 2.
%
% However, this file is supposed to be a template to be modified
% for your own needs. For this reason, if you use this file as a
% template and not specifically distribute it as part of a another
% package/program, I grant the extra permission to freely copy and
% modify this file as you see fit and even to delete this copyright
% notice. 

\documentclass{beamer}
\usepackage[utf8]{inputenc}
\usepackage[brazilian]{babel}

% There are many different themes available for Beamer. A comprehensive
% list with examples is given here:
% http://deic.uab.es/~iblanes/beamer_gallery/index_by_theme.html
% You can uncomment the themes below if you would like to use a different
% one:
%\usetheme{AnnArbor}
%\usetheme{Antibes}
%\usetheme{Bergen}
%\usetheme{Berkeley}
%\usetheme{Berlin}
%\usetheme{Boadilla}
%\usetheme{boxes}
\usetheme{CambridgeUS}
%\usetheme{Copenhagen}
%\usetheme{Darmstadt}
%\usetheme{default}
%\usetheme{Frankfurt}
%\usetheme{Goettingen}
%\usetheme{Hannover}
%\usetheme{Ilmenau}
%\usetheme{JuanLesPins}
%\usetheme{Luebeck}
%\usetheme{Madrid}
%\usetheme{Malmoe}
%\usetheme{Marburg}
%\usetheme{Montpellier}
%\usetheme{PaloAlto}
%\usetheme{Pittsburgh}
%\usetheme{Rochester}
%\usetheme{Singapore}
%\usetheme{Szeged}
%\usetheme{Warsaw}

\title{Aula 4 - Unix}

% A subtitle is optional and this may be deleted
\subtitle{Curso de Unix}

\author{PET Computa\c{c}ão}
% - Give the names in the same order as the appear in the paper.
% - Use the \inst{?} command only if the authors have different
%   affiliation.

\institute[UFSC] % (optional, but mostly needed)
{
%
  Departamento de Informática e Estatística\\
  Universidade de Santa Catarina}
% - Use the \inst command only if there are several affiliations.
% - Keep it simple, no one is interested in your street address.

\date{PET Computa\c{c}ão, 2015}
% - Either use conference name or its abbreviation.
% - Not really informative to the audience, more for people (including
%   yourself) who are reading the slides online

\subject{Curso de Unix}
% This is only inserted into the PDF information catalog. Can be left
% out. 

% If you have a file called "university-logo-filename.xxx", where xxx
% is a graphic format that can be processed by latex or pdflatex,
% resp., then you can add a logo as follows:

% \pgfdeclareimage[height=0.5cm]{university-logo}{university-logo-filename}
% \logo{\pgfuseimage{university-logo}}

% Delete this, if you do not want the table of contents to pop up at
% the beginning of each subsection:
\AtBeginSubsection[]
{
  \begin{frame}<beamer>{Sumário}
    \tableofcontents[currentsection,currentsubsection]
  \end{frame}
}

% Let's get started
\begin{document}

\begin{frame}
  \titlepage
\end{frame}

\begin{frame}{Sumário}
  \tableofcontents
  % You might wish to add the option [pausesections]
\end{frame}

% Section and subsections will appear in the presentation overview
% and table of contents.
\section{Truques do Linux}

\subsection{Caractere coringa}

\begin{frame}{Truques do Linux}{Caractere coringa}
  \begin{itemize}
  \item {
   O caractere \textbf{*} é considerado um caractere coringa no linux, ele é utilizado quando não sabemos especificar quantos e quais caracteres desejamos na pesquisa, por exemplo: \textbf{ls *.jpg}, este comando irá listar todos os arquivos \textbf{.jpg} dentro do diretório atual independente do nome.
  }
  \item{Outro simbolo importante é o \textbf{?}, utilizamos ele quando não sabemos espeficicar somente um caractere, por exemplo:
  \textbf{ls ?ato.jpg} este comando poderá ter retornos do tipo: \textbf{gato.jpg},\textbf{ pato.jpg},\textbf{ mato.jpg}.}
 \end{itemize}
\end{frame}

\section{Boas práticas}
\subsection{Boas práticas de nomea\c{c}ão de arquivos}
\begin{frame}{Boas práticas}{Boas práticas de nomea\c{c}ão de arquivos}
  \begin{itemize}
  \item {
   Ao nomearmos arquivos ou diretórios, devemos evitar a utiliza\c{c}ão de caracteres especiais, os arquivos devem ser nomeados com letras e números, espa\c{c}os também devem ser evitados e como alternativa devemos colocar o \textbf{\_} para melhor leitura. 
  }
  \item{Além do nome dos arquivos, devemos também colocar sua extensão, extensões dizem ao sistema operacional como ler aquele tipo de arquivo.}
 \end{itemize}
 \end{frame}
 
 \begin{frame}{Boas práticas}{Boas práticas de nomea\c{c}ão de arquivos}
   \begin{center}
 \begin{tabular}{||c | c||} 
 \hline
 \textbf{Más práticas} & \textbf{Boas práticas}\\ [0.5ex] 
 \hline\hline
 Meu Arquivo Lindo.txt & meu\_arquivo\_lindo.txt\\ 
 \hline
 projeto & projeto.c\\
 \hline
 Pre\c{c}os \& datas.txt & precos\_e\_datas.txt\\
 \hline
\end{tabular}
\end{center}
\end{frame}

\section{Manuais on-line e descri\c{c}ões dos comandos}
\subsection{man e apropos}
\begin{frame}{Manuais on-line e descri\c{c}ões dos comandos}{man e apropos}
\begin{itemize}
\item {Ao digitar \textbf{man} \textit{comando}, conseguimos o manual completo do comando com todas as flags e explica\c{c}ões necessárias para a utiliza\c{c}ão correta do mesmo.}
\item{Quando usamos \textbf{apropos} \textit{palavra\_chave} obtemos como retorno os comandos que possuem aquela palavra chave no header do seu manual.}
\item{Outra ferramenta interessante é o \textbf{whatis} \textit{comando}, ele nos retornará uma descri\c{c}ão sucinta do comando especificado. }
\end{itemize}
\end{frame}

% Placing a * after \section means it will not show in the
% outline or table of contents.
\section*{Sumário dos comandos}

\begin{frame}{Sumário dos comandos}
 \begin{center}
 \begin{tabular}{||c | p{9cm}||} 
 \hline
 \textbf{Comando} & \textbf{Descri\c{c}ão}\\ [0.5ex] 
 \hline\hline
 man & recebemos o manual do comando especificado\\ 
 \hline
 apropos & recebemos os comandos que possuem a palavra\_chave em seu manual\\
 \hline
 whatis & descri\c{c}ão sucinta do comando especificado\\
 \hline
\end{tabular}
\end{center}

 \begin{center}
 \begin{tabular}{||c | p{9cm}||} 
 \hline
 \textbf{Caractere} & \textbf{Descri\c{c}ão}\\ [0.5ex] 
 \hline\hline
 * & caractere coringa que representa qualquer ou nenhuma sequência de caracteres\\ 
 \hline
 ? & caractere coringa que representa um carcater qualquer\\
 \hline
\end{tabular}
\end{center}
\end{frame}

\end{document}


